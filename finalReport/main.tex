\documentclass{article} % For LaTeX2e
\usepackage{nips15submit_e,times}
\usepackage{hyperref, dsfont}
\usepackage{url}
%\documentstyle[nips14submit_09,times,art10]{article} % For LaTeX 2.09


\title{Larp - Laplace Noise and Multiple Additive Regression Trees}


\author{Peter Rindal 
\And
Trung Viet Vu 
\And
Hung Viet Le}

% The \author macro works with any number of authors. There are two commands
% used to separate the names and addresses of multiple authors: \And and \AND.
%
% Using \And between authors leaves it to \LaTeX{} to determine where to break
% the lines. Using \AND forces a linebreak at that point. So, if \LaTeX{}
% puts 3 of 4 authors names on the first line, and the last on the second
% line, try using \AND instead of \And before the third author name.

\newcommand{\fix}{\marginpar{FIX}}
\newcommand{\new}{\marginpar{NEW}}

\nipsfinalcopy % Uncomment for camera-ready version

\begin{document}

 
\maketitle

\begin{abstract}

Boosting is a successful machine learning algorithm for a number of regression and classification tasks. Combined with tree base learner, boosted ensemble algorithms are shown to deliver highly accurate and strong predictive models. In this work, we first study a well-known approach called Multiple Additive Regression Trees (MART). By looking at its weakness, we proposed a different approach named Laplace Additive Regression Trees (LART) to mitigate this by injecting random Laplace noises. In the experiment, we compare the performance of LART and MART on the dataset that has been used in previous papers. Our results is promising as the proposed method yields significant gains in accuracy in this particular task.

\end{abstract}

\section{Introduction}

% introduce the problem. See the dart paper for a good intro. We want to frame it as the MART algorithm suffers from later trees not contributing much and that the consecutive trees have too much in common. i.e. the same predicates are often chosen between consecutive tree when the learning rate/shrinkage is small , e.g. 0.05. 

Ensemble modeling is a powerful way to improve the performance of a single learning model.  It can be able to out-perform any single classifier within the ensemble \cite{Dietterich2000}. Many methods for constructing ensembles have been developed. One popular approach is Bagging and Random Forest \cite{Breiman2001}, in which different subsets of the training examples are used to learn independent predictors. Another practical technique is boosting. Boosted algorithms focus on improving the current model by iteratively adjusting the learning algorithm between iterations. In general, this method often results in better reducing the bias error and achieving strong predictive models. \\

In 2001, Friedmn \cite{mart} introduced an ensemble of boosted regression trees - MART, which combines boosting with tree base learner. At each iteration a tree is added to fit the gradient using least square. The resulting model forms an ensemble of weak predictive models that gives highly accurate predictions. Later on, Caruana and Niculescu-Mizail \cite{Caruana06anempirical} demonstrated the application of MART to build highly successful models for many learning tasks. Despite its success, this algorithm still suffers from the problem of over-fitting where the learned model does not generalize well to unseen data. This issue is addressed in \cite{VinayakG15} as \textit{over-specialization}: trees added at later iterations tend to have too much in common, and add negligible contribution towards the prediction of all the remaining instances. In real-world settings, the MART algorithm often starts with a single tree that learns the bias of the problem while the additive trees in the ensemble learn the deviation from this bias. In this sense, the ensemble is sensitive to the decision made by the first tree.\\

One commonly used tool to approach the problem of over-specialization in MART is \textit{shrinkage} \cite{mart}. Empirically this regularization technique reduces the impact of each tree by a constant value - \textit{shrinkage factor}, which helps dramatically improve the model's generalization ability. However, while the over-fitting problem can be mitigated by shrinkage, the fundamental issue of over-specialization still remains, i.e. as the size of the ensemble increases, the contribution of the later trees is negligible \cite{VinayakG15}. Motivated by this, we hypothesize that randomizing the process of selecting predicates can provide another effective regularization for MART and propose the LART algorithm.


\subsection{Mart - Random Forest Hybrid}

% maybe one of you can write this. Talk about how our randomization technique allows a smooth transition between a random forest technique and the original MART. This can be tuned and talk about how the use of the laplace random variable effects this randomization. I.e. if one predicate/split is significantly better, then it will still be pick with good probability (assuming a moderate amount of noise.)

In this work, we explore an improvement in MART to address the issue of over-specialization. Specifically, we propose a method called Laplace Additive Regression Trees (LART), which injects randomness into the learning algorithm.
By adding Laplace noises to the calculation of the loss function, a predicate can be sampled in a way such that a greater reduce in the loss function results in it being exponentially more likely to be selected. Instead of selecting the predicate that minimizes the loss function, we select the predicate that minimizes the sum of the loss function and the Laplace noise. Predicates chosen in this way are more uncorrelated and hence the variance in the learners increases.\\

Our motivation for this approach is a hybrid view between MART and Random Forest. While MART takes the advantage of boosting algorithms in achieving low bias and low variance, it can be sensitive to noise and outliers as mentioned above. On the other hand, Random Forest benefits from the diversity among the learned classifiers and hence it is more robust to noise and outliers. However, it has the major disadvantage of regression trees, that is, their inaccuracy. Our proposed method tries to capture the advantage of these two approaches with appropriate additive Laplace noise. When the Laplace distribution is scaled by too much, the noise will drown out the contribution that the loss function may have. Predicates at each iteration are likely to be chosen randomly. This leads to special case wherein LART is similar to a Random Forest algorithm. Conversely, if the Laplace noise is too small, the loss function will dominate and effectively result in the MART algorithm. Tuning the scale of Laplace noise thus is a major point of investigation for our work.



\subsection{Differential Privacy }

An alternative way to interpret this work is from the perspective of differential privacy. The general setting of differential privacy is that we wish to run an algorithm on some sensitive dataset and reveal the result. Moreover, this result must have privacy guarantees with respect to how much information on \emph{any single record} can be obtained from the result alone. 

While this work is not concerned about privacy, we are concerned with the problem of over fitting which decision trees are particularly susceptible to. Overfitting is a general problem in machine learning where the model learns random errors in the training set as opposed to the underlying distribution. 

In more detail, differential privacy guarantees that for two datasets $X,Y$ where $Y$ has one training example $X$ does not, then the probability that the training algorithm on $X$ and $Y$ will produce the same models should be bounded by
$$
	\forall m \in\mbox{Models} \ : \ \Pr[m = \textsc{Train}(X)] \leq \exp(\epsilon) \Pr[m= \textsc{Train}(Y)]
$$
What this definition says is that the algorithm \textsc{Train} should output the model $m$ with roughly the same probability when the parameter $\epsilon\in (0,\infty]$ is near $0$. In the case that $\epsilon$ is very large, this definition of differential privacy places no constraint the the output of the algorithm. In the setting of differential privacy, a smaller value for $\epsilon$ is considered more private, e.g. $\epsilon=0.1$. 

Comparing this to the problem of overfitting, we can see many parallels. That is, if $Y = X \cup {y}$ and $y$ is an training example with which significantly deviates from the underlying distribution, then a training algorithm which is differentially privacy will give guarantees about the impact that this outlier can have. This observation is the primary inspiration of this for this work. In particular, we do not try to fully satisfy the requirements of differential privacy, instead we apply differentially private techniques in a heuristically manner.

\section{Related Work}    
 
 Gradient boosting is a general technique to build a linear combination of based learners that starts with an initial guessed learner and in each step, adds a single tree to the ensemble learned in the previous step in a linear way to minimize the lost function.  Friedman et al.\cite{mart} adapted the gradient boosting technique to build MART, an ensemble of tree base learners. It has been reported\cite{Caruana06anempirical} to produce highly accurate predictions. But, its main weakness is \textit{over-specialization}:  trees added at later iterations tend to have too much in common, and add negligible contribution towards the prediction of all the remaining instances. We will get back to this problem in more details in Section~\ref{sec:over-spe}. Rashmi and Gilad-Bachrach proposed addressing this problem by using randomization. Specifically, they randomly drop a subset of based learners learned in previous steps out of the current ensemble to train the next ensemble. We propose a different approach of utilizing randomization that incorporate Laplace noise to overcome over-specialization.
 
Regarding injecting randomness into an ensemble of trees, there are roughly two ways~\cite{PV07} in literature: \emph{input randomization} and \emph{feature randomization}. Two extrem representatives of two classes are Bagging~\cite{Breiman96} and Extra-Tree~\cite{GEW06} where Bagging only employs input randomization and Extra-Tree only employs feature randomization.

In input randomization, the original idea is to subsample the training set to obtain $B$ training subsets where each training subset is sampled (with replacement) according to the uniform distribution.  Each training subset is used to train a decision tree and then, trees are aggregated together into a single predictor, thereby reducing the noise effect. Wagging~\cite{BK98} is another variant of input randomization method where each sample is assigned a random weight and all training samples are retained in the training set. Since Bagging can be seen as Wagging where the weights come from discrete Poisson distribution, Webb et al.~\cite{WZ04} proposed assigning weights following continuous Poisson distribution.  

Beside input randomization, there are vast majority of works built upon the idea of feature randomization. Ho~\cite{Ho98} proposed sampling a subset of features, called a subspace, projecting training data on to this subspace and building a decision tree for the projected training data. Thus, each subspace sampling would give a tree and a collection of subspace samples gives a forest. Another idea is to inject randomization in the splitting process while growing the tree. Dietterich~\cite{Dietterich00} proposed an approach that  randomly choose an attribute from a set of potential candidate attributes to split. There could be some flexibility in defining a set of potential candidate attributes and the randomization can also be added to the split value itself~\cite{CZ01,GEW06}. Also, one can use a random linear combination~\cite{Breiman2001} of attributes to determine the split threshold. Our work lies in this (feature randomization) category.
\section{Over Specialization} \label{sec:over-spe}

Over-specialization of MART was identified  by Rashmi and Gilad-Bachrach~\cite{dart} in their work on DART. Essentially, they observed that the added last trees tend to have impact on a small number of instances. As a result, the initial added trees would have a huge contribution to the ensemble. Note that the based tree learners are very simple function of the input and parameters. This causes the negative effect on the performance of the resulting ensemble on unseen data. Though over-specialization can be reduced by \emph{shrinkage}, i.e, multiplying leaf values of each to-be-added tree by a constant value in $(0,1)$, it still persists.  Rashmi and Gilad-Bachrach demonstrated their observation through extensive experiments and argued that over-specialization can be eliminated using randomization.     In Section~\ref{sec:algo}, we introduce a different randomized scheme to overcome this issue.

\section{The Algorithm} \label{sec:algo}

\subsection{Mart}
We begin by describing the Mart algorithm on which we will build the Lart algorithm. The Mart algorithm can be viewed as a (functional) gradient decent algorithm \cite{friedman2001}. The model is then constructed iteratively by generating a series of trees, each correcting the previous set. This is done by taking the derivative of the loss function based the current predictions and updates the model by adding another regression tree that is trained to fit the negation of these derivative. 

The input to the algorithm is a labeled dataset $D=\{(x,y)\}$ where $x\in \mathcal{X}$ is a vector in the feature space $\mathcal{X}$ and $y\in \mathcal{Y}$ is the labels in prediction space. In addition to $D$, the algorithm takes as input a loss function $\mathcal{L}_{(x,y)} :  \mathcal{Y} \mapsto \mathds{R} $ which the models prediction to a real value denoting the error of this prediction. In our case, we are performing regression and chose to use the $L2$ loss function and therefore we have $\mathcal{L}_{(x,y)}(\hat{y} ) = (y- \hat{y})^2$.
 
 
The algorithm begins by initializing the current model $M:\mathcal{X}\mapsto \mathcal{Y}$ to some default map. Let $M(x)$ denote the prediction of the model at the feature vector $x$ and $\mathcal{L}'_{(x,y)}(M(x))$ denote the derivative of the loss function at $M(x)$. At each iteration, Mart then creates new dataset datasets $D'=\{(x,-\mathcal{L}'(M(x)))\}$  and learns a new tree model $T$ for $D'$. This tree $T$ is trained to predict the negation of the the derivate of the loss function. The current model is then updated as $M(x):= M(x) + T(x)$, thereby reducing the loss.

In the case of the $L2$ loss function describe above, its derivative is $\mathcal{L}'_{(x,y)}(\hat{y}) = 2(y - \hat{y})$. This will be the loss function of choice in the evaluation section as it is suitable to the regression task in question. However, other loss functions can be defines for different tasks such as classification. In this case, the  the logistic loss function can be used $\mathcal{L}_{(x,y)} (\hat{y}) =( 1 + \exp(\lambda y \hat{y}))^{-1}$ and consecutive models will be trained to fit its derivative.

{\color{red} add a section about shrinkage}


\subsection{Lart}
We now turn our attention to the Lart technique discussed in this paper. One problem with the Mart algorithm is that of over specialization as discussed above\cite{dart}. In this case the later trees contribute very little to the  model as trees start to fail to fit the derivative of the loss function. The related work Dart \cite{dart} suggest that randomizing how these trees are constructed can provide some benefit in that later trees continue to improve the model. While they put forth a technique based on dropout, our technique is based on methods from differential privacy.


First let turn our attention to how an individual tree $T$ is trained. It is assumed that there exists a set of $m$ predicates each of which $P:\mathcal{X} \mapsto \{0,1\}$ map a feature vector to a single bit.  Initially, all records in the training set $D'$ are mapped to the root of the tree. The tree is then recursively constructed by selecting a lead node $L$. For each predicate $P_i$ the recodes $(x,y)\in L$ are mapped to $L$'s newly created ``candidate" children $L_{i,0},L_{i,1}$, where $(x,y)$ is mapped to $L_{i,P_i(x)}$. For each of these candidate nodes $L_{i,b}$, the label $\hat{y}_{i,b}$ that minimizes the loss function $\mathcal{L}$ when applied $(x,y)\in L_{i,b}$ is computed. The pair of leaf nodes $L_{i,0}, L_{i,1}$ which has the greatest reduction in the loss function over $L$ is then selected and $L_{i,0}, L_{i,1}$ are added to the tree as leaf nodes. This process of selecting and splitting leaf nodes continues until some stop condition it reached, e.g. the best $k$ leaves have been split.

The Lart algorithm randomizes this process of selecting predicates in the following way. Instead of strictly selecting the leaf node pair $L_{i,0}, L_{i,1}$ with the greatest reduction in the loss function, we noisily select the best predicate. In particularly, for each predicate we sample a random variable $\eta\gets \mathcal{D}$ and computed the its effective reduction in the loss function as $\ell + \eta$, where $\ell$ was the original reduction. The predicate that has the greatest ``effective" reduction in the loss function is then chosen.

The choice of the distribution $\mathcal{D}$ can therefore have a significant impact on the performance of the model. In the case that $\ell \ll \eta$ with high probability,  the Larp algorithm is identical to a random forest with an appropriate choice of when to stop the algorithm. However, in the case that $\ell \gg \eta$, the Larp algorithm is identical to the original Mart algorithm. As such, this modification to the training process allow us to explore the the space of algorithms between Mart and random forest.

In our experimental evaluation where we use the $L2$ loss function we chose to instantiate this distribution $\mathcal{D}$ as the Laplace distribution centered at zero and scaled by $\frac{\mu}{\epsilon}$, where $\mu$ is the mean value of the labels at the leaf and $\epsilon$ is a parameter that allows the amount of noise to be tuned. This choice was made to on average the amount of noise added to an reduction is roughly proportional to the effect of adding/removing a single record from the node in question. Moreover, by choosing to use the Laplace distribution with this scaling, we are approximating the requirements for differential privacy which we had previous interpreted as a potential mechanism for reducing overfitting/over-specialization. However, other distributions such as a Gaussian would likely have a similar effect. 


\section{Evaluation}

We evaluate the our Lart technique by comparing it to the Mart, random forests, and Dart\cite{dart} algorithms. We implemented each of their algorithms within our own framework. While Dart\cite{dart} also implemented implemented these algorithms, the numbers we get are slightly different than what they report. 

For the dataset, we choose to use the CT slice dataset from \cite{graf20112d} which can be found at \cite{uci_ctSlice}. This datset contains 53500 histograms created from CT scans of 74 individuals. The task is to locate the position that the scan was taken. Each image is represented as 386 features. We used 10-fold cross validation to compare our algorithm. As with the prior work that we compare to \cite{dart}, we ensure that the splits are chosen such that each individual is completely in the test or training datasets.


\begin{figure}	\centering
	\begin{tabular}{|r|}
 \hline
 
 \hline
               RF \\
               MART \\
               Lart \\\hline
	\end{tabular} 	
	\caption{ }
\end{figure}



Table \ref{tab:compare} show the performance of the Larp technique as compared Mart, Dart, and random forest.

\section{Conclusions}
In this project, we first review the approach of using Multiple Additive Regression Trees (MART) to regression and classification tasks. The weakness of MART is trees added at later iterations have significantly diminishing contributions. By injecting some randomness to generating ensembles classifiers, we propose a different approach, called LART, to provide efficient regularization for MART. Our experiment shows that LART out-performs MART when the number of trees increases considerably. This strengthen our hypothesis that LART is more robust to over-specialization.\\

This study also suggests some point of investigation. One direction is tuning the scale parameter along with other traditional parameters in MART and examining their interaction. Additionally, we might evaluate the performance of LART on more complicated datasets and other machine learning tasks, as in data mining.



\nocite{*}
\bibliographystyle{plain}
\bibliography{ref}


\end{document}
